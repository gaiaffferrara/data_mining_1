
% Template Tesi in Informatica, ispirata al template "Tesi di laurea - Universit� di Pisa" di Simone Schirinzi (https://www.overleaf.com/latex/templates/tesi-di-laurea-universita-di-pisa/rwdcqtqwftpg). 

% Carattere dimensione 12
\documentclass[12pt]{report}
\usepackage{caption}  % For customizing captions
\usepackage{booktabs} % For better looking tables


% Per la stampa fronte-retro sostituire con:
% \documentclass[12pt, twoside]{report}

% Margini (4cm a sx, 2.5cm a dx, 2.5cm in alto, 2.5cm in basso)
\usepackage[top=2.5cm, bottom=2.5cm, left=2.5cm, right=2.5cm, centering]{geometry}

% Per la stampa fronte-retro sostituire con: 
% \usepackage[top=2.5cm, bottom=2.5cm, inner=4cm, outer=4cm, right=2.5cm, centering]{geometry}

% Interlinea
\linespread{1.3}


% Librerie utili
%\usepackage[italian]{babel} % applicazione regole di scrittura per la lingua italiana
\usepackage[english]{babel} % applicazione regole di scrittura per la lingua inglese 
\usepackage[utf8]{inputenc} % codifica UTF-8
\usepackage{scrlayer-scrpage} % stili pagina per il frontespizio
\ifoot[]{}
\cfoot[]{}
\ofoot[\pagemark]{\pagemark}
\pagestyle{scrplain}


%\usepackage{mathptmx} % font Times New Roman (simile)
%\renewcommand{\rmdefault}{fontname} %cambiare font
\renewcommand{\sfdefault}{fontname} % Sans-serif
%\renewcommand{\ttdefault}{fontname} % Monospace



\usepackage{graphicx} % inserimento di immagini
\usepackage{csquotes} % per le citazioni "in blocco"
%\usepackage[backend=biber, sorting=nty, ]{biblatex} % bibliografia con pacchetto biblatex (https://ctan.org/pkg/biblatex?lang=en)
\appto{\bibsetup}{\raggedright}

\usepackage{titlesec} % per la formattazione dei titoli delle sezioni, capitoli etc.
\usepackage{float} % per il posizionamento delle immagini

\usepackage{listings} % per il codice di programmazione
% Fonte https://en.wikibooks.org/wiki/LaTeX/Source_Code_Listings. Per la lista di sintassi riconosciute.
\renewcommand{\lstlistingname}{Code}% Listing -> Codice
\usepackage{xcolor}  % stile del codice
\definecolor{mygreen}{rgb}{0,0.6,0}
\definecolor{mygray}{rgb}{0.5,0.5,0.5}
\definecolor{mymauve}{rgb}{0.58,0,0.82}
\definecolor{darkgray}{rgb}{.4,.4,.4}
\definecolor{navy}{HTML}{000080}
\definecolor{purple}{rgb}{0.65, 0.12, 0.82}
\definecolor{codepurple}{rgb}{0.58,0,0.82}
\definecolor{backcolour}{rgb}{0.95,0.95,0.92}

% Stili configurabili del codice (lslisting) 
\lstset{ %
belowcaptionskip=0.5em,
backgroundcolor=\color{backcolour}, % choose the background color; you must add \usepackage{color} or \usepackage{xcolor}
basicstyle=\footnotesize, % the size of the fonts that are used for the code
breakatwhitespace=false, % sets if automatic breaks should only happen at whitespace
breaklines=true, % sets automatic line breaking
captionpos=b, % sets the caption-position to bottom
commentstyle=\color{mygreen}, % comment style
deletekeywords={...}, % if you want to delete keywords from the given language
escapeinside={\%*}{*)}, % if you want to add LaTeX within your code
extendedchars=true, % lets you use non-ASCII characters; for 8-bits encodings only, does not work with UTF-8
frame=single, % adds a frame around the code
keepspaces=true, % keeps spaces in text, useful for keeping indentation of code (possibly needs columns=flexible)
keywordstyle=\color{codepurple}, % keyword style
% language=Octave, % the language of the code
morekeywords={*,...}, % if you want to add more keywords to the set
numbers=left, % where to put the line-numbers; possible values are (none, left, right)
numbersep=5pt, % how far the line-numbers are from the code
numberstyle=\tiny\color{mygray}, % the style that is used for the line-numbers
rulecolor=\color{black}, % if not set, the frame-color may be changed on line-breaks within not-black text (e.g. comments (green here))
showspaces=false, % show spaces everywhere adding particular underscores; it overrides 'showstringspaces'
showstringspaces=false, % underline spaces within strings only
showtabs=false, % show tabs within strings adding particular underscores
stepnumber=1, % the step between two line-numbers. If it's 1, each line will be numbered
stringstyle=\color{mymauve}, % string literal style
tabsize=2, % sets default tabsize to 2 spaces
title=\lstname % show the filename of files included with \lstinputlisting; also try caption instead of title
}

% END of listing package 

% Esempio riconoscimento sintassi JavaScript 

\lstdefinelanguage{JavaScript}{
  keywords={typeof, new, true, false, catch, function, return, null, catch, switch, var, const, let, if, in, while, do, else, case, break},
  keywordstyle=\color{blue}\bfseries,
  ndkeywords={class, export, boolean, throw, implements, import, this},
  ndkeywordstyle=\color{darkgray}\bfseries,
  identifierstyle=\color{black},
  sensitive=false,
  comment=[l]{//},
  morecomment=[s]{/*}{*/},
  commentstyle=\color{mygreen}\ttfamily,
  stringstyle=\color{red}\ttfamily,
  morestring=[b]',
  morestring=[b]"
}

\lstset{
   language=JavaScript,
   extendedchars=true,
   basicstyle=\footnotesize\ttfamily,
   showstringspaces=false,
   showspaces=false,
   numbers=left,
   numberstyle=\footnotesize,
   numbersep=9pt,
   tabsize=2,
   breaklines=true,
   showtabs=false,
   captionpos=b
}

% Formato delle intestazioni
\titleformat{\chapter}[block]
  {\normalfont\LARGE\bfseries}{\thechapter.}{0.5em}{\LARGE}
\titlespacing*{\chapter}{0pt}{-20pt}{25pt}



\begin{document}

% Frontespizio
\begin{titlepage}
% \begin{figure}
%     \centering\includegraphics[scale=0.5]{immagini/cherubino_pant541.png}
% \end{figure}

\begin{center} %SISTEMARE MARGINI 
    % {\LARGE{ Data Science and Business Informatics Master Course \\}}
    % \vspace{2cm}
    {\LARGE { Data Mining: Foundations }}\\
    \vspace{2cm}
    {\Large { Group 12 }}\\
    \vspace{3cm}
    {\large { Bruno Barbieri, Noemi Dalmasso, Gaia Federica Francesca Ferrara }}
\end{center}

% \vspace{2cm}

% \begin{minipage}[t]{0.47\textwidth}
%   {\large{\bf Relatore:\\ Nome Cognome}}
% 	\vspace{0.5cm}
% 	{\large{\bf \\Correlatore:\\ Nome Cognome}}
% \end{minipage}\hfill\begin{minipage}[t]{0.47\textwidth}\raggedleft
%   {\large{\bf Group 12\\ }}
%   \vspace{0.5cm}
  
% \end{minipage}

% \vspace{25mm}

% \centering{\large{\bf ANNO ACCADEMICO 20xx/20xx }}
\end{titlepage}
% Fine frontespizio

\tableofcontents
\thispagestyle{empty}

%\listoffigures ho tolto questo così l'introduzione sta subito dopo l'indice
%\thispagestyle{empty}

\clearpage
\setcounter{page}{1}
\addtocontents{toc}{\protect\thispagestyle{empty}}
\addcontentsline{toc}{chapter}{Introduction} % Capitolo non numerato
\section*{Introduction}
\label{ch:introduzione}

The aim of this report is to display an analysis carried out on the IMDb dataset, which contains data about movies, TV shows, 
and other forms of visual entertainment, along with their ratings generated by the internet community. 
The analysis has been conducted making use of data mining methodologies. After the data understanding and preparation phase, 
clustering, classification, and pattern mining techniques have been applied.


\clearpage

\chapter{Data Understanding and Preparation}
\label{ch:capitolo1}

% --- Inizio del Capitolo 1 ---

% Capitolo 1...

% Esempio di una nota\footnote{CleanCode} con citazione a sezione~\ref{sec:sezione1}.

% \begin{figure}[H]
%     \centering
%     \includegraphics[width=0.2\textwidth]{immagini/logo-dip_blu_hr.png}
%     \caption{Esempio di un'immagine}
%     \label{fig:immagine1}
% \end{figure}

% \begin{lstlisting}[caption={Esempio codice JavaScript},label={Esempio codice JavaScript}, language=JavaScript]
% function foo(num) {
%     const bar = 2;
%     return num + bar;
% }

% const result = foo(2);
% // result -> 4
% \end{lstlisting}


% \section{Sezione 1}\label{sec:sezione1}


% Sezione 1... con citazione bibliografica~\cite{CleanCode}

% \subsection{Sottosezione 1}
% \label{subsec:sottosezione1}

% Sottosezione 1...

\section{Data Semantics}\label{sec:data_semantics}
The dataset \textit{complete\_df.csv}, which is the merge of the original \textit{train.csv} and \textit{test.csv} datasets, contains 21909 titles of different forms of visual entertainment that have been rated on IMDb, an online database of information related to films, television series etc. 
Each record is described by 23 attributes, both numerical and non-numerical. 
All the variables of the dataset are introduced and explained in Table 1.1 and Table 1.2.
\begin{table}[h]
    \centering
    \begin{tabular}{|l|l|l|} % Using 'l' for left alignment of columns
        \hline
        \textbf{Attribute} & \textbf{Type} & \textbf{Description} \\ 
        \hline
        originalTitle & Nominal & Title in its original language \\  
        \hline
        rating & Ordinal & IMDB title rating class \\
        & & The range is from (0,1] to (9,10] \\ 
        \hline
        titleType & Nominal & The format of the title \\ 
        \hline
        canHaveEpisodes & Nominal (Binary) & Whether or not the title can have episodes \\ 
        & & True: can have episodes; False: cannot have episodes \\ 
        \hline
        isRatable & Nominal (Binary) & Whether or not the title can be rated by users \\ 
        & & True: it can be rated; False: cannot be rated \\ 
        \hline
        isAdult & Nominal (Binary) & Whether or not the title is for adults \\ 
        & & 0: non-adult title; 1: adult title \\ 
        \hline
        countryOfOrigin & Nominal & The country(ies) where the title was produced \\ 
        \hline
        genres & Nominal & The genre(s) associated with the title \\ 
        \hline
    \end{tabular}
    \caption{Description of non-numerical attributes}
    \label{tab:attributes}
\end{table}
\begin{table}[h]
    \centering
    \begin{tabular}{|l|l|l|} % Using 'l' for left alignment of columns
        \hline
        \textbf{Attribute} & \textbf{Type} & \textbf{Description} \\ 
        \hline
        runtimeMinutes & Numeric & Runtime of the title expressed in minutes \\ 
        \hline
        startYear & Interval & Release/start year of a title \\ 
        \hline
        endYear & Interval & TV Series end year \\
        \hline
        awardWins & Numeric & Number of awards the title won \\ 
        \hline
        numVotes & Numeric & Number of votes the title has received \\ 
        \hline
        worstRating & Numeric & Worst title rating \\ 
        \hline
        bestRating & Numeric & Best title rating \\ 
        \hline
        totalImages & Numeric & Number of Images on the IMDb title page \\ 
        \hline
        totalVideos & Numeric & Number of Videos on the IMDb title page \\ 
        \hline
        totalCredits & Numeric & Number of Credits for the title \\ 
        \hline
        criticReviewsTotal & Numeric & Total Number of Critic Reviews \\ 
        \hline
        awardNominationsExcludeWins & Numeric & Number of award nominations excluding wins \\ 
        \hline
        numRegions & Numeric & The regions number for this version of the title \\ 
        \hline
        userReviewsTotal & Numeric & Number of User Reviews \\ 
        \hline
        ratingCount & Numeric & The total number of user ratings for the title \\ 
        \hline
    \end{tabular}
    \caption{Description of numerical attributes}
    \label{tab:numerical_attributes}
\end{table}
\section{Distribution of the variables and statistics}\label{sec:variable_distrib}
\subsection{Discrete attributes}
...content...
...content...
...content...
...content...
...content...
...content...
...content...
...content...
...content...
...content...

\subsection{Continuous attributes}
...content...
...content...
...content...
...content...
...content...
...content...
...content...
...content...
...content...
...content...


\end{document}


\clearpage

\addcontentsline{toc}{chapter}{Conclusioni} % Capitolo non numerato
\chapter*{Conclusioni}
\label{ch:conclusioni}

Conclusioni...

\bibliographystyle{plain} % We choose the "plain" reference style
\bibliography{bibliography} % Entries are in the refs.bib file
%\addbibresource{bibliography.bib}
%\printbibliography

\end{document}
